\documentclass{tufte-handout}

\usepackage{amsmath}

\begin{document}

  \title{Fixed Effects Interacting Across Two Categories}
  \author{Liu, Minsheng}
  \date{April 3rd, 2018}
  \maketitle

  \section{Introduction}

  Here we discuss fixed effects interacting across two categories.

  For instance, consider Alice, Bob, and Cathy work for Macrohard and Redhole.
  Their performance can be linearly related to the multiplication of some known
  factor of each individual, such as age, and some unknown factor of each
  company.

  Some notation. We use $(A, F)$ to denote some factor $F$ of some category $A$.
  We use $C = ((A, F), (B, G))$ to denote a fixed effect interacting across
  two categories $A$ and $B$ through $F$ and $G$, where $F$ is known and $G$
  is to estimate. Thus, a model concerned here looks like
  \[
    Y = X \beta + \langle F, G \rangle + \epsilon \, ,
  \]
  where $\langle \cdot, \cdot \rangle$ is defined as follows.
  For each observation (row) $i$, with $m \in A$ and $n \in B$ being
  the row's corresponding individual in those two categories, we have
  \[
    \langle F, G \rangle_i = F_m G_n \, .
  \]
  Clearly, given any dataset $\phi$, we can represent $\langle F, G \rangle$
  by matrix multiplication, $F_\phi G_\phi$.
  As a concrete example, consider the following fabricated data:

  \begin{center}
    \begin{tabular}{l | c} \hline
      Individual & Age \\ \hline
      Alice & 21 \\
      Bob & 24 \\
      Cathy & 27 \\ \hline
      Company & Factor \\ \hline
      Macrohard & 0.3 \\
      Redhole & 0.4 \\ \hline
    \end{tabular}
  \end{center}

  If our dataset is in the order of AM, BM, CM, AR, BR, CR, we have
  \[
    F_{\phi} =
      \begin{bmatrix}
        21 & 0 \\
        24 & 0 \\
        27 & 0 \\
        0 & 21 \\
        0 & 24 \\
        0 & 27
      \end{bmatrix}
  \]
  and
  \[
    G_{\phi} =
      \begin{bmatrix}
        0.3 \\ 0.4
      \end{bmatrix} \, .
  \]

  \section{Solution}

  We solve general models of this kind by applying the
  Frisch--Waugh--Lovell theorem. In other words, we project everything
  but those fixed effects interacting across two categories onto the
  orthogonal complement of the projection matrix $P_F = F(F^\top F)^{-1}F^\top$,
  or equivalently, $1 - P$.

  For multiple fixed effects of this kind, instead of MAP, we choose to apply
  FWL in iterations. For instance, consider
  \[ Y = X \beta + F_1G_1 + F_2G_2 + F_3G_3 + \epsilon \, . \]
  We have
  \begin{align*}
    M_3Y &= M_3X \beta + (M_3F_1)G_1 + (M_3F_2)G_2 + \epsilon \\
    M_2M_3Y &= M_2M_3X \beta + (M_2M_3F_1)G_1 + \epsilon \\
    M_1M_2M_3Y &= M_1M_2M_3X \beta + \epsilon \, ,
  \end{align*}
  where
  \begin{align*}
    M_3 &= 1 - P_{F_3}\\
    M_2 &= 1 - P_{M_3 F_2} \\
    M_1 &= 1 - P_{M_2M_3 F_1} \, .
  \end{align*}
  There are two benefits not to use MAP here. First, the original reason we use 
  demean and MAP is to avoid solving a very large matrix. Since for fixed
  effects discussed here, the projection has to be solved explicitly anyway,
  FWL seems to be the way to go. Second, in this manner, estimation can be easy.

  Estimation can be done as follows. Consider first the system
  \[ Y = X \beta + F G + \epsilon \, . \]
  After $\beta$ is estimated, we can subtract $X \beta$ on both sides to get
  \[ Y - X \beta = F G + \epsilon \, . \]
  OLS can be used to estimate for $G$. Since $F(F^\top F)^{-1}F^\top$
  has already been computed, the process does not involve too much 
  computation overhead. The algorithm extends easily to systems with multiple
  fixed effects.

  If our system further includes simple fixed effects, we would estimate
  them first, demeaning both $X$ and fixed effects interacting across two 
  categories. After that, everything is the same as before.
  The existing algorithm to estimate those simple fixed effects still works,
  but we need to remove the effect of demeaning not only $X$ but also other
  fixed effects interacting across two categories.

\end{document}
